\section{Considerations}

\subsection{Disadvantages}

\begin{itemize}
      \item Firstly, I would like to point out the limitation of the data used. Even though
            it is not the problem with the overall approach, it is worth mentioning. In
            particular, the dataset has the genre imbalance as can be seen on
            figure~\ref{fig:dataanalysis:genre_distr}, which is pretty important issue, but
            is not addressed in this work (in data preprocession, but consider
            section~\ref{training-process}).
      \item One of the main limitations of \textit{GNN} is the inability of distinguishing
            the different structures of the graphs. This issue is related to the graph
            isomorphism. It is hard to say whether this concrete limitation affecting the
            problem we are trying to solve. However, since we are considering graph
            structures, we need to take into consideration every possible issue. Moreover,
            one of the possible solutions if \textit{Graph Isomorphism
                  Network}~\cite{xuHowPowerfulAre2019}
      \item I do not see any point in considering noise vulnerability of the \textit{GNN}s,
            since this issue is clearly not related to the problem considered.
      \item Furthermore, \textit{GNN}s suffer from time-space complexity issues. It is very
            difficult to scale such a network. Number of edges in dense graph can reach up
            to \(O(n^2)\). Therefore, all the computations difficulty and space
            requirenments raise drastically.
\end{itemize}

\subsection{Advantages}

\begin{itemize}
      \item If the data represented in the natural for it form (i.e.~graphs), it allows to
            capture difficult connections between the data nodes. Learning from not the
            data itself, but the connections as well, allows for better accuracy and
            overally better results.
      \item Moreover, natural representation of the data increases the interpretability of
            the processes behind the scenes, as well as, the interpretability of the final
            results.
      \item The overall generality of the approach allows to apply the theory to a wide
            variety of the problems. Since many real-world structures can be respresented
            as graphs, different interpretations are possible.
            \begin{enumerate}
                  \item Convolution
                  \item Attention
                  \item LSTM
                  \item RNN
                  \item CV
                  \item and many others
            \end{enumerate}
\end{itemize}
